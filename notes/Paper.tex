\documentclass[prd,preprintnumbers,twocolumn,eqsecnum,floatfix,letter]{revtex4}
\usepackage{color}
\usepackage{calc}
\usepackage{amsmath,amssymb,graphicx}
\usepackage{amssymb,amsmath}
\usepackage{tensor}
\usepackage{bm}
\usepackage{times}
\usepackage[varg]{txfonts}
\usepackage[colorlinks, pdfborder={0 0 0}]{hyperref}
\definecolor{LinkColor}{rgb}{0.75, 0, 0}
\definecolor{CiteColor}{rgb}{0, 0.5, 0.5}
\definecolor{UrlColor}{rgb}{0, 0, 0.75}
\hypersetup{linkcolor=LinkColor}
\hypersetup{citecolor=CiteColor}
\hypersetup{urlcolor=UrlColor}
\maxdeadcycles=1000
\allowdisplaybreaks
\textwidth 7.5 in 
\hoffset -1 cm 
\newcommand{\comment}[1]{\textcolor{blue}{\textit{#1}}}
\newcommand{\ashok}[1]{\textcolor{red}{\textit{Ashok:#1}}}
\newcommand{\sean}[1]{\textcolor{cyan}{\textit{Sean:#1}}}


\begin{document}

\newcommand{\be}{\begin{equation}}
\newcommand{\ee}{\end{equation}}
\newcommand{\ber}{\begin{eqnarray}}
\newcommand{\eer}{\end{eqnarray}}
\def\bea{\begin{eqnarray}}
\def\eea{\end{eqnarray}}
\newcommand{\etal}{\emph{et al.}}

\newcommand{\Sl}{S_\ell}
\newcommand{\Sigmal}{\Sigma_\ell}
\newcommand{\Flux}{\mathcal{F}}
\newcommand{\LNh}{\hat{\mathbf{L}}_N}
\newcommand{\LN}{\mathbf{L}_N}
\newcommand{\bS}{\mathbf{S}}
\newcommand{\bJ}{\mathbf{J}}
\newcommand{\e}{\mathrm{e}}
\newcommand{\rmi}{\mathrm{i}}
\newcommand{\flow}{f_\mathrm{low}}
\newcommand{\fcut}{f_\mathrm{cut}}

\newcommand{\bchi}{\bm{\chi}}
\newcommand{\blambda}{\bm{\lambda}}
\newcommand{\bLambda}{\bm{\Lambda}}
\newcommand{\bchia}{\bm{\chi}_a}
\newcommand{\bchis}{\bm{\chi}_s}
\newcommand{\chis}{\chi_s}
%\newcommand{\bchi}{\mathbf{\chi}}
\newcommand{\chia}{\chi_a}
\newcommand{\chiadL}{\bchia \cdot \LNh}
\newcommand{\chisdL}{\bchis \cdot \LNh}
\newcommand{\chisSqr}{\bchis^2}
\newcommand{\chiaSqr}{\bchia^2}
\newcommand{\chisDchia}{\bchis \cdot \bchia}
\newcommand{\cA}{\mathcal{A}}
\newcommand{\cB}{\mathcal{B}}
\newcommand{\cC}{\mathcal{C}}
\newcommand{\cP}{\mathcal{P}}
\newcommand{\pc}{{+,\times}}

\title{Spin effect on Christodoulou memory for merging blackhole binaries}
\author{Ashok Choudhary}\email{aschoudhary@mix.wvu.edu}
\author{Sean T. McWilliams}\email{sean.mcwilliams@mail.wvu.edu}
\affiliation{Department of Physics and Astronomy, West Virginia University, Morgantown, WV 26506, USA}


\begin{abstract}
	The Christodoulou memory, which is a nonlinear memory effect sourced by the gravitational wave stress tensor, produce a growing, nonoscillatory change in the gravitational wave "plus" polarization. This results in the permanent displacement of a pair of freely falling test masses after the gravitational wave has passed. While the memory contribution during early in-spiral is well described by Post-Newtonian approximation, the contribution at merger must be obtained from Numerical relativity simulations. The Post-Newtonian corrections upto 3PN order to the gravitational wave memory for quasicircular, inspiralling compact binaries for non spinning has been computed by Favata et al.[Physical Review D 80, 024002 (2009)]. Here we include the Spin contribution at leading order and calculate the memory effect upto 3PN order. We also compute memory contribution at merger from Numerical relativity using Weyl curvature component $\psi_4$ after removing the unphysical low frequency contribution due to spectral leakage. We use the method described in [Class. Quantum Grav. 28 195015] for integration which control the amplification of the unphysical frequency resulting from spectral leakage. 
\end{abstract}

\maketitle

\section{Introduction}
Gravitational wave memory effect leads to a permanent displacement of test masses after the gravitational wave has passed through it. Nonlinear memory effect also called Christodoulou memory, arises from a change in the radiative multiple moment that is sourced by the energy flux of the radiated gravitational wave. It can be understood in the following way. Consider the Einstein field equation in harmonic gauge. [add ref] 
\begin{subequations}
\begin{align}	
\Box \bar{h}^{\alpha \beta} & = -16\pi (-g)(T^{\alpha\beta} + t_{LL}^{\alpha\beta})-\bar{h}^{\alpha\mu}_{,\nu}\bar{h}^{\beta\nu}_{,\mu}+\bar{h}^{\mu\nu}\bar{h}^{\alpha\beta}_{,\mu\nu}\\
\bar{h}^{\alpha \beta},_{\beta} & = 0
\end{align}
\end{subequations}
where
\begin{equation}
\bar{h}^{\alpha \beta} = \eta^{\alpha\beta}-\sqrt{-g}g^{\alpha\beta} 
\end{equation}
is the gravitational field tensor, $g$ is the determinant of the metric $g_{\alpha\beta}$, $T^{\alpha\beta}$ is the matter stress-energy tensor, $t^{\alpha\beta}_{LL}$ is the Landau-Lifshitz pseudotensor, $\Box \equiv  -\partial^{2}_{t} + \nabla^{2}$ is the flat-space wave operator, a comma denotes a partial derivative $(,_{\mu}\equiv \partial_{\mu})$, and $\nabla^{2}$ is a flat-space Laplace operator. The Landau-Lifshitz term is the Gravitational wave stress tensor[add ref].
\begin{equation}
	T^{gw}_{jk}=\frac{1}{32\pi}\langle h^{TT}_{ab,j}h^{TT}_{ab,k}\rangle\approx T^{gw}_{00}n_{j}n_{k} = \frac{1}{R^{2}}\frac{dE^{gw}}{dt d\Omega}n_{j}n_{k}
\end{equation}
where $\frac{dE^{gw}}{dtd\Omega}$ is the GW energy flux, $n_{j}$ is a unit radial vector, the angle-bracket mean to average over several wavelengths, $h^{TT}_{ab,j}=\bar{h}^{TT}_{ab,j}$. When we apply standard Green's function to right hand side of equation (1.1a), we obtain the following correction term to GW field [add ref]
\begin{equation}
	\delta h^{TT}_{jk} = \frac{4}{R}\int_{-\infty}^{T_{R}} dt'\Bigg[ \frac{dE^{gw}}{dt'd\Omega'}\frac{n'_{j}n'_{k}}{(1-\boldsymbol{n'.N})}d\Omega'\Bigg]^{TT}
\end{equation}
where $T_{R}$ is the retarded time. This equation shows that part of the distant GW is sourced by loss of GW energy.



\section{POST-NEWTONIAN WAVEFORM POLARIZATION AND MODE DECOMPOSITION}
Following Kidder [add ref] we introduce a asymptotically-flat radiative coordinate system described by spherical coordinates $(T, R, \Theta, \Phi)$ and their orthogonal basis vectors $(\vec{e}_T, \vec{e}_R, \vec{e}_{\Theta}, \vec{e}_{\Phi})$. The coordinate origin is center-of-mass of the source. The retarded time in radiative coordinates is $T_R = T - R$
\\
The mode decomposition of gravitational wave polarization is given by
\begin{equation}
	h_+ - \mathit{i}h_{\times} = \sum_{l=2}^{\infty}\sum_{m=-l}^{m=l}h^{lm} \,  _{-2}Y^{lm}(\Theta, \Phi)
\end{equation}
where
\begin{equation}
	h^{lm}= \frac{G}{\sqrt{2}\, R \, c^{l+2}}\left[U^{lm}(T_R)-\frac{\mathit{i}}{c}V^{lm}(T_R)\right]
\end{equation}

Here $U^{lm}(T_R)$ and $V^{lm}(T_R)$ are "scalar" mass and current multipoles respectively. The spin weighted spherical harmonics are defined in terms of the Wigner $d$ function by
\begin{equation}
	_{-2}Y^{lm}(\Theta, \Phi) = (-1)^2\sqrt{\frac{2l + 1}{4\pi}}d^{l}_{ms}(\Theta)e^{\mathit{i}m\Phi}
\end{equation} 
Here
\begin{align}
	d^{l}_{ms}&=\sqrt{(l+m)!(l-m)!(l+s)!(l-s)!}\\
	&\times \sum_{k=k_i}^{k_f}\frac{(-1)^k(\sin\frac{\Theta}{2})^{2k+s-1}(\cos\frac{\Theta}{2})^{1l+m-s-2k}}{k!(l+m-k)!(l-s-k)!(s-m+k)!}
\end{align}
where $k_i$ = max$(0, m-s)$ and $k_f$ = min$(l+m, l-s)$. 

\section{MEMORY CONTRIBUTION TO THE RADIATIVE-MASS MULTIPOLES}
The GW polarization can be expanded on the basis of spin-weighted spherical harmonics.
\begin{equation}
	\sqrt{2}R(h_{+} - ih_{\times}) = \sum_{l=2}^{\infty}\sum_{m=-l}^{l}(U_{lm} - \mathit{i}V_{lm})_2Y^{lm}(\Theta, \Phi)
\end{equation}
Here $(R,\Theta,\Phi)$ are the spherical coordinates pointing from the source to the observer. The coefficient $U_{lm}$ and $V_{lm}$ are mass and current multipole moments. \\
The memory contribution to the waveform polarization can be evaluated in the the following ways as described in [add referece]
 
\subsection{Computing angular integral}


\begin{equation}
	\int d\Omega n_{L},
\end{equation}

\begin{equation}
	\int d\Omega _{-2}Y_{l'm'}(\theta,\phi)_{-2}Y_{l''m''}^{*}(\theta, \phi)Y_{lm}^{*}(\theta, \phi)
\end{equation}

\begin{align}
	U_{lm}^{(mem)(1)}=&R^{2}\sqrt{\frac{2(l-2)!}{(l+2)!}}\sum_{l'=2}^{\infty}\sum_{l''=2}^{\infty}\sum_{m'=-l'}^{l'}\sum_{m''=-l''}^{l''}(-1)^{m+m'}\\
	&\times\Bigg\langle \dot{h}_{l'm'}\dot{h}_{l'm'}^{*}\Bigg\rangle G^{2-20}_{l'm'lm'-m''-m}
\end{align}

\subsection{Time derivatives of the memory multipole moments}

\begin{equation}
	\int_{-\infty}^{T_{R}} x^{n}dt=\int_{-\infty}^{T_{R}}\frac{x^{n}}{\dot{x}}dx
\end{equation}

\subsection{Computing time integral over the past history of the source}
sesadsd
 
% 
\begin{widetext}
\begin{align}\label{eq:FbydE}
\frac{dx}{dt} & = \frac{64\, x^5  \, \eta}{5} \left\{1 + x \left[-\frac{743}{336}-\frac{11 \eta }{4} \right] 
    + x^{3/2} \left[4 \pi-\frac{113}{12}\delta \, \chiadL- \left(-\frac{113}{19} + \frac{19 \, \eta}{3}  \right)\chisdL \right] \right. \nonumber \\ 
    &\quad + x^2 \left[ \frac{34103}{18144} + \frac{13661 \, \eta}{2016} + \frac{59 \, \eta^2}{18} + \left( \frac{719}{96}- 30 \eta\right)(\chiadL)^2 +\chiaSqr \left(-\frac{233}{96}+10 \eta \right) \right. \nonumber\\
   &\qquad+\frac{719 \chiadL \, \chisdL \delta
   }{48}-\frac{233 \chisDchia \delta}{48} \left. + \, (\chisdL)^2 \left(-\frac{\eta }{24}-\frac{719}{96}\right)+\chisSqr \left(\frac{7 \eta
   }{24}+\frac{233}{96}\right) \right] \nonumber \\ 
    &\quad + x^{5/2} \left[-\frac{4159 \, \pi}{672}-\frac{189 \, \pi \, \eta}{8} + \chiadL \delta  \left(-\frac{31319}{1008} + \frac{1159 \, \eta}{24}\right)+\chisdL \left(-\frac{31319}{1008} + \frac{22975 \, \eta}{252} -\frac{79 \, \eta^2}{3}\right) \right]  \nonumber\\ 
    &\quad \left. + x^3 \left[\frac{16447322263}{139708800} + \frac{16 \, \pi^2}{3} -\frac{1712 \, \gamma_{E}}{105} -\frac{56198689 \, \eta}{217728} + \frac{451 \, \pi^2 \, \eta}{48} + \frac{541 \, \eta^2}{896} -\frac{5605 \, \eta^{3}}{2592} -\frac{80 \,\pi \delta\chiadL }{3} \right.\right. \nonumber \\
    &\qquad \left.\left. + \left(\frac{575}{448} + \frac{565 \, \delta^2}{9}-\frac{65815 \, \eta}{4032} + \frac{89 \, \eta^2}{2}\right)(\chiadL)^2 + \left(-\frac{145}{448}+\frac{21985 \,\eta}{4032} -\frac{89 \, \eta^2}{6}\right)\chiaSqr + \right(-\frac{80 \, \pi}{3} + \frac{40 \, \pi \, \eta}{3} \right. \nonumber \\
    &\qquad  \left.+\delta\left(\frac{258295}{2016} -\frac{9203 \, \eta}{96}\right)\chiadL\left)\chisdL + \left(\frac{258295}{4032} -\frac{48773 \, \eta}{576} + \frac{3041 \, \eta^2}{144}\right)(\chisdL)^2 + \delta\left(-\frac{145}{224} + \frac{2143 \, \eta}{288}\right)\chisDchia \right.\right.\nonumber \\ 
    &\qquad\left.\left.+\left(-\frac{145}{448} + \frac{1891 \, \eta}{576} -\frac{7 \, \eta^2}{144}\right)\chisSqr - \frac{3424 \, \log[2]}{105} - \frac{1712 \log[x]}{210}\right]\right\}
\end{align}
\end{widetext}


\subsection{RESULTS: MEMORY CONTRIBUTION TO THE POST-NEWTONIAN WAVEFORM OF QUASI-CIRCULAR, INSPIRALLING BINARIES}
The memory contribution to the spin-weighted spherical-harmonic modes of the polarization waveform [ref eqn]. These quantities are related via 
\begin{equation}
	h_{l0}^{(mem)} = \frac{\alpha}{\sqrt{2} \, R} U_{l0}^{(mem)} = 8 \sqrt{\frac{\pi}{5}}\frac{\eta M x}{R} 
\end{equation}
where we have followed the notation of Sec. 9 of Ref [add referece]. The notational parameter $\alpha$ accounts for the two commonly used choices for the polarization triad 
\begin{equation}
	content...
\end{equation}
(\ashok{Need to add}).\\
The results of polarization modes in terms of $\hat{H}_{l0}$ are:
\begin{widetext}
\begin{align}\label{eq:H20}
\hat{H}_{20} &= \alpha \frac{5}{15\sqrt{6}}\left\{ 1 + x\left(-\frac{4075}{4032} + \frac{67 \, \eta}{48}\right)+ x^{3/2}\left(\left(-\frac{7}{1440} -\frac{27 \, \delta}{10} + \frac{7 \,\eta}{360}\right)\chiadL+\left(-\frac{27}{10} -\frac{7 \, \delta^3}{1440} + \frac{22 \, \eta}{15}\right)\chisdL\right)\right. \nonumber \\
& \quad \left. x^2\left(-\frac{151877213}{67060224} -\frac{123815 \, \eta}{44352} +\frac{205 \, \eta^2}{352} -\left(\frac{1459}{864} -\frac{1441 \, \eta}{216}\right)\chiaSqr - \left(\frac{27 \, \delta}{8} +\frac{\delta^3}{432}\right)\chisDchia - \left(\frac{27}{16}+\frac{\delta^4}{864} -\frac{\eta}{12}\right)\chisSqr\right)\right. \nonumber \\
& \quad \left. x^{5/2}\left(-\frac{253 \, \pi}{336}+ \frac{253 \, \pi \, \eta}{84} -\left(-\frac{631}{96768} + \frac{2729857 \, \delta}{84672} -\frac{146597 \, \pi \, \delta}{1764}+\frac{5 \, \eta}{378}+\frac{3007 \, \delta \, \eta}{1008} -4\pi \, \delta \, \eta +\frac{311 \, \eta^2}{6048}\right)\chiadL\right.\right. \nonumber\\
& \qquad\left. -\left(\frac{2729857}{84672}-\frac{146597 \, \pi}{1764}-\frac{631 \, \delta^3 }{96798} -\frac{44111 \, \eta}{1764}+\frac{4852 \, \pi \, \eta}{63}-\frac{311 \, \delta^3 \, \eta}{24192}-\frac{1121 \, \eta^2}{252} +\frac{68 \, \pi \, \eta^2}{7}\right)\chisdL\right) \nonumber \\
& \quad  x^3\left[-\frac{4397711103307}{532580106240} + \left( \frac{700464542023}{13948526592} - \frac{205 \, \pi^2}{96}\right)\eta + \frac{69527951 \, \eta^2}{166053888} + \frac{1321981 \, \eta^3}{5930496} -  \left(-\frac{160867}{13824} -\frac{791 \, \delta}{27648} + \right.\right. \nonumber \\
& \qquad \left. \frac{7345 \, \delta^2}{576} + \left(\frac{3953767}{96768}+\frac{791 \, \delta}{6912}\right)\eta + \frac{22885 \, \eta^2}{1152}\right)\chiaSqr -\left(-\frac{275 \, \pi}{12} + \frac{7 \, \pi \, \delta^3}{768} + 17 \, \pi \, \eta \right)\chisdL + \left(\frac{47869}{43008}  \right.\nonumber\\
& \qquad \left.\left. - \frac{791 \, \delta^3}{27648} + \frac{743 \, \delta^4}{387072} + \left(-\frac{410453}{16128} + \frac{133 \, \delta^3}{6912} + \frac{11 \, \delta^4}{4608}\right)\eta + \frac{10229 \, \eta^2}{1152}\right)\chisSqr - \left( \frac{7 \, \pi}{768}-\frac{275 \, \pi \, \delta}{12}-\frac{7 \, \pi \, \eta}{192}\right.\right. \nonumber \\
&\qquad \left.\left.\left. + \left(-\frac{791}{27648} + \frac{47869 \, \delta}{21504} + \frac{743 \, \delta^3}{193536} - \frac{791 \, \delta^4}{27648} + \left(\frac{77}{576} - \frac{71749 \, \delta}{2304} + \frac{11 \, \delta^3}{2304}\right)\eta - \frac{133 \, \eta^2}{1728}\right)\chisdL\right)\chisdL
\right]
\right\}
\end{align}

\end{widetext}

\section{Memory calculation using Numerical relativity data}
Numerical relativity simulations most commonly output the Newman-Penrose curvature component $\psi_4$.The two polarization states, $h_{+}$ and $h_{\times}$ of the gravitational wave are related to the curvature, expressed in terms 	of the complex Newman-Penrose scalar $\psi_4$ by 
\begin{equation}
	\psi_4 = \ddot{h}_+ - \mathit{i}\ddot{h}_{\times}
\end{equation}\
Given the Newman-Penrose scalar $\psi_4$ for a particular mode, we have to integrate twice in time to obtain $h_{+}$ and $h_{\times}$. It has long been noted that producing a strain, $h$, from the Newman-Penrose curvature component, $\psi_4$, typically results in a waveform with unphysical secular non-linear drift. The nonlinearity of drift indicates that this is not simply a result of two constants of integration involved in transformation. This nonlinearity is potentially caused by the fact that $\psi_4$ is typically extracted at a finite distance from the gravitating source. The strain $h$ is only related to $\psi_4$ at an infinite distance and also strictly valid in a particular gauge. As a result in numerical simulation, the finite distance calculation introduces a systematic error. In more recent simulations [added ref], which can possibly extract truly gauge-invariant waveform at future null infinity has not able able to get rid of the secular nonlinear drift.
\par It has been argued in [add ref] that an important source of unphysical non-linear drift in numerical computation of gravitational wave strain lies in the transformation of the measured data to the strain $h$  which generally involves an integration in time. The output of the numerical simulation is a discretely sampled time series of finite duration, incorporating some component of unresolved frequencies due to numerical error. This can lead to an uncontrollable non-linear drift if the integration is performed in the time domain. 
\par The memory contribution to $h_{+}$ is given by 

\begin{equation}
	h_{+}^{(mem)} \approx \frac{R}{192 \, \pi}s_{\Theta}^{2}\left(17 +c_{\Theta}^{2} \right)\int_{-\infty}^{T_R}|\dot{h}_{22}|^2 dt
\end{equation}

The above equation requires a time integration of absolute value of $\dot{h}_{22}$, but this is not typically the quantity which is directly computed in numerical simulation. The output of numerical simulation is usually in the form of component of curvature tensor components, or Zerilli-Moncrief-type variables defined relative to a background. 
\subsection{Evaluating Bondi News from numerical data}
The Bondi News is determined by
\begin{equation}
	\mathcal{N} = \int_{-\infty}^{t}dt' \psi_4
\end{equation}

\subsection{Integration of finite length signals in frequency domain}

Consider the Fourier transform, $\mathcal{F}$, applied to an absolutely integrable function $f(t)$,
\begin{equation}
	\tilde{f}(\omega) = \mathcal{F}[f]=\int_{-\infty}^{\infty}\e^{-\iota \omega t}f(t)dt.	
\end{equation} 
The Fourier transform of the time integral is given by
\begin{equation}
	\mathcal{F}\left[\int_{-\infty}^{t}dt' f(t')\right] |_\omega = -\iota \frac{\tilde{f}(\omega)}{\omega}
\end{equation}

\subsubsection{Fixed frequency integration ($\mathit{FFI}$)}

We evaluate the integral using the following mathod [add ref]



\[
\tilde{F} =
\begin{cases}
-\mathit{i} \frac{\tilde{f}(\omega)}{\omega_0} & \text{ $\omega\leq\omega_0$} \\
-\mathit{i} \frac{\tilde{f}(\omega)}{\omega} & \text{ $\omega>\omega_0$} 

\end{cases}
\]
%%%%%%%%%%%%%%%%%%%%%%%%%%%%%%%%%%%%%%%%%%%%%%%%%%%%%%%%%
\begin{figure}
\includegraphics[width=4.0in]{../plots/MemoryPlot_nonSpining/q7.pdf}
\caption{NonSpinning case}
\label{fig:q7}
\end{figure}

\begin{figure}
\includegraphics[width=4.0in]{../plots/MemoryPlot_alignedSpin/0p95.pdf}
\caption{NonSpinning case}
\label{fig:0p95}
\end{figure}

\newpage
\appendix 


\bibliography{Spin}

\end{document}
